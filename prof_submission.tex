\documentclass[11pt,a4paper]{article}
\usepackage[margin=1in]{geometry}
\usepackage{amsmath,amssymb}
\usepackage{hyperref}
\usepackage{enumitem}
\usepackage{kotex}
\setlist[enumerate]{leftmargin=*,itemsep=0.3em,topsep=0.4em}
\setlist[itemize]{leftmargin=*}

\title{Budgeted Causal Repair for Reliable Multi-Hop RAG\\\large (연구 제안 요약본)}
\author{김병준}
\date{\today}

\begin{document}
\maketitle

\section*{한 줄 요약}
멀티홉 RAG에서 오류 가능성이 높은 hop만 선별적으로 반사실 검증하고 국소 복구하며, 이를 예산 제약 최적화 문제로 정식화하는 연구.

\begin{abstract}
멀티홉 RAG는 복합 질의에서 높은 잠재력을 보이지만, 초기 hop의 근거 누락과 관계 단절이 후속 hop으로 전파되어 최종 정답과 근거 신뢰성을 동시에 저하시킨다. 기존 연구는 검색 강화, 오류 위치화, 부분 재생성, 리스크 추정 중 일부 축에 집중하며, 어떤 hop에 몇 번의 반사실적 검증을 배분할지 예산 제약 하에서 최적화하는 문제를 직접 다루지 않는다. 본 연구는 Budgeted Causal Repair (BCR)를 제안한다. BCR은 (i) 저비용 게이팅으로 의심 hop 후보를 추출하고, (ii) 후보 hop에만 링크 단위 개입을 수행해 LCIS(Link Causal Influence Score)를 추정하며, (iii) 순차검정 기반 조기종료와 hop별 예산 할당으로 복구 범위를 동적으로 결정한다. 복구는 재검색, 링크 재연결, 부분 재생성으로 구성된다. HotpotQA, 2WikiMultiHopQA, MuSiQue, MultiHop-RAG에서 EM/F1, evidence coverage, omission rate, link-consistency, error localization F1, cost-per-gain을 평가한다. 본 연구는 동일 또는 유사 예산 하에서 정확도와 근거 체인 신뢰성을 동시에 개선하는 멀티홉 RAG 운용 방법론을 제시한다.
\end{abstract}

\section{연구 메타}
\textbf{문제 한 줄 정의:} 멀티홉 RAG에서 오류는 hop 간 연결을 따라 전파되며, 이를 전수 검증하면 비용이 과도하고 무검증 복구는 신뢰성이 낮다.

\textbf{핵심 가설}
\begin{enumerate}
  \item 오류 가능 hop만 선별해 반사실적 개입을 수행하면 동일 예산에서 더 높은 정확도와 신뢰성을 달성할 수 있다.
  \item LCIS 기반 복구 우선순위가 heuristic 기반 복구보다 cost-per-gain이 우수하다.
  \item 순차검정 조기종료가 성능 저하 없이 counterfactual 호출비를 줄인다.
\end{enumerate}

\section{연구 배경과 공백}
\begin{enumerate}
  \item 멀티홉 QA는 단계별 근거 연결 실패가 성능 하락의 핵심 원인이다.
  \item 기존 멀티홉 RAG 계열은 검색 경로 품질을 높이지만, 실패 단계의 인과적 기여도를 정책적으로 활용하는 경우가 드물다.
  \item 오류 위치화 계열은 실패 지점 탐지에 강하지만, 검증 호출비를 최적화 대상으로 다루지 않는 경우가 많다.
  \item 본 연구의 공백은 \textbf{오류 위치화 + 인과 점수 + 예산 최적화}를 단일 복구 정책으로 결합하는 데 있다.
\end{enumerate}

\section{연구 목표와 질문}
\textbf{목표:} 예산 제약 하에서 정확도(EM/F1), 근거 체인 품질, 비용 효율을 동시에 개선하는 복구 정책 확립.

\textbf{연구질문}
\begin{enumerate}
  \item LCIS는 복구 우선순위의 유효한 결정 신호인가?
  \item 순차검정 조기종료는 비용을 줄이면서 성능을 유지하는가?
  \item 동일 budget tier에서 BCR은 기존 방법 대비 Pareto 우위를 보이는가?
\end{enumerate}

\section{방법: Budgeted Causal Repair (BCR)}
\subsection{문제 정식화}
\begin{equation*}
\begin{gathered}
\max_{\pi}\; \mathbb{E}[U(y)] \quad \text{s.t.}\quad \mathbb{E}[C(\pi)] \le B \\
U(y)=\alpha_1\cdot \text{AnsScore}(y)+\alpha_2\cdot \text{EvidenceCoverage}(y)-\alpha_3\cdot \text{Omission}(y)-\alpha_4\cdot \text{LinkInconsistency}(y) \\
C(\pi)=c_{\text{tok}}N_{\text{tok}}+c_{\text{ret}}N_{\text{ret}} \\
\hspace{2.2em}+c_{\text{rep}}N_{\text{rep}}+c_{\text{lat}}T_{\text{lat}}
\end{gathered}
\end{equation*}

\begin{itemize}
  \item AnsScore는 추론 시 verifier proxy, 평가 시 EM/F1로 대체.
  \item $B$는 질의당 허용 비용(토큰/호출/지연) 예산.
\end{itemize}

\subsection{단계 1: 저비용 게이트}
모든 hop를 검증하지 않고 후보 $K$만 추출한다.
\begin{enumerate}
  \item 인용 문장 수 부족 또는 동일 인용 반복
  \item hop $t$의 핵심 엔티티가 hop $t+1$에서 연결되지 않음
  \item self-consistency 저하
  \item verifier-lite 점수 하위 분위수
\end{enumerate}
\begin{align}
K=\{h_t \mid g(h_t)\ge \tau_g\}
\end{align}

\subsection{단계 2: 링크 단위 반사실적 개입과 LCIS}
\textbf{개입 타입}
\begin{enumerate}
  \item Bridge intervention: bridge entity/관계 제거 또는 치환
  \item Relation-only intervention: 엔티티 고정, 관계 표현만 치환
  \item Citation swap: claim 유지, citation만 교체
\end{enumerate}

\begin{align}
\widehat{LCIS}_t=\frac{1}{n_t}\sum_{i=1}^{n_t}\left(S(y)-S\left(y^{(i,t)}\right)\right)
\end{align}
\begin{itemize}
  \item $S(y)$는 $[0,1]$로 정규화된 품질 점수로 정의한다:
  \item $S(y)=\lambda_1\widetilde{\text{AnsScore}}(y)+\lambda_2\widetilde{\text{EvidenceCoverage}}(y)-\lambda_3\widetilde{\text{Omission}}(y)-\lambda_4\widetilde{\text{LinkInconsistency}}(y)$, $\sum_k\lambda_k=1$.
  \item $\widehat{LCIS}_t$가 클수록 해당 hop 개입의 성능 영향이 큰 것으로 해석한다.
\end{itemize}

\subsection{단계 3: 순차검정과 조기종료}
각 후보 hop에 대해 최소 샘플 $n_{\min}$부터 시작한다.
\begin{enumerate}
  \item $CI_{\text{lower}}(LCIS_t)>\tau_{\text{high}}$: 중요한 hop로 확정, 복구 후보 유지, 테스트 중단
  \item $CI_{\text{upper}}(LCIS_t)<\tau_{\text{low}}$: 비중요 hop로 확정, 후보 제외, 테스트 중단
  \item 그 외: $n_t<n_{\max}$이고 예산이 남으면 추가 테스트
\end{enumerate}

\subsection{단계 4: 예산 기반 복구 우선순위}
\begin{align}
R_t=\beta_1\cdot Risk_t+\beta_2\cdot \widehat{LCIS}_t-\beta_3\cdot \widehat{Cost}_t+\beta_4\cdot UncertaintyBonus_t
\end{align}
\begin{itemize}
  \item 해석 일관성을 위해 $R_t$는 ``높은 위험 + 높은 인과영향 + 낮은 비용'' hop를 우선한다.
\end{itemize}

복구 실행 조건:
\begin{align}
R_t>\eta,\quad C_{\text{used}}+\Delta C_t\le B
\end{align}

복구 연산:
\begin{enumerate}
  \item 재검색 (retrieve again with focused query)
  \item 링크 재연결 (re-link entities/relations)
  \item 부분 재생성 (local regenerate from failing hop)
  \item 전체 재시도(full rerun)는 최후 수단
\end{enumerate}

\subsection{알고리즘 요약}
\begin{enumerate}
  \item Hop 분해 및 게이팅으로 후보 추출
  \item 후보별 LCIS 순차추정
  \item 우선순위 점수 $R_t$ 계산
  \item 예산 내 복구 반복
  \item 예산 소진 또는 추가 이득 미미 시 종료
\end{enumerate}

\section{구현 상세 계획}
\subsection{시스템 모듈}
\begin{enumerate}
  \item HopParser: hop 분해 및 링크 추출
  \item GateScorer: 저비용 위험 점수 계산
  \item InterventionEngine: 3종 개입 생성
  \item LCISEstimator: 순차추정 및 신뢰구간 업데이트
  \item BudgetAllocator: 잔여 예산 기반 우선순위 실행
  \item RepairExecutor: 재검색/재연결/부분재생성 수행
  \item Evaluator: 정답/근거/비용 지표 계산
\end{enumerate}

\subsection{권장 하이퍼파라미터 시작점}
\begin{enumerate}
  \item $\tau_g$: 상위 30\% 위험 hop 통과
  \item $n_{\min}=2,\; n_{\max}=6$
  \item $\tau_{\text{low}}=0.02,\; \tau_{\text{high}}=0.10$
  \item budget tier: B0/B1/B2/B3 = 0/+10\%/+20\%/+30\%
\end{enumerate}

\subsection{계산복잡도}
질의당 추가 호출량은 대략 $O(|K|\cdot \bar n)$이며, 게이팅이 유효하면 $|K| \ll T$가 되어 전체 비용을 억제할 수 있다.

\section{실험 설계(재현 중심)}
\subsection{데이터셋}
\begin{enumerate}
  \item HotpotQA
  \item 2WikiMultiHopQA
  \item MuSiQue
  \item MultiHop-RAG benchmark
\end{enumerate}

\subsection{비교군}
\begin{enumerate}
  \item Vanilla RAG
  \item Adaptive-RAG
  \item HopRAG
  \item PropRAG
  \item HydraRAG
  \item ERL (재현 가능 시)
\end{enumerate}

\subsection{공정성 통제}
\begin{enumerate}
  \item 동일 backbone LLM
  \item 동일 retriever/index/top-k/context budget
  \item 동일 decoding
  \item seed 5회 반복
  \item 동일 budget tier에서만 비교
\end{enumerate}

\subsection{비용 동등 비교 페어링 규칙}
\begin{enumerate}
  \item pair 단위는 dataset/method pair/budget tier/seed로 고정한다.
  \item 비용 동등 조건은 $\;0.95 \le C_{\text{method}}/C_{\text{ref}} \le 1.05\;$이다.
  \item $\text{ref}$는 비교쌍의 기준 방법으로 고정한다 (예: BCR-Full vs Vanilla에서는 Vanilla, BCR-Full vs Full-Rerun에서는 Full-Rerun).
  \item 비용 비동등 pair는 주가설 검정에서 제외하고 부록 민감도 분석으로 분리 보고한다.
\end{enumerate}

\subsection{평가 지표}
\begin{enumerate}
  \item 정답: EM, F1
  \item 근거 품질: Evidence Coverage, Omission Rate, Link-Consistency
  \item 위치화: Error Localization F1
  \item 효율: 토큰, retrieval calls, latency, retry count
  \item 종합: Cost-per-Gain, Pareto Frontier
\end{enumerate}
\begin{itemize}
  \item Cost-per-Gain$=\dfrac{C_{\text{method}}-C_{\text{baseline}}}{\Delta F1}$로 두되, $\Delta F1 \le 0$인 경우는 ``CPG-Invalid''로 태깅하고 win-rate 집계에서 제외한다.
\end{itemize}

\subsection{통계}
\begin{enumerate}
  \item Primary endpoint: B2(+20\%)의 cost-matched macro $\Delta F1$
  \item paired bootstrap 10{,}000회
  \item 95\% CI 보고
  \item 주요 비교쌍 유의성 검증 (paired permutation 병행)
  \item EM 개선은 secondary endpoint로 별도 보고
\end{enumerate}

\section{아블레이션 설계}
\begin{enumerate}
  \item Full BCR
  \item No-Gate
  \item No-LCIS (heuristic score만 사용)
  \item No-Sequential-Test (고정 $n$)
  \item No-Budget-Allocation (균등 배분)
  \item Random-Hop-Repair
  \item Full-Rerun-Repair
\end{enumerate}

추가 분석:
\begin{enumerate}
  \item hop 길이별(2/3/4+) 성능 변화
  \item 오류 유형별(omission/drift/relational) 개선폭
  \item 예산 민감도(B0~B3) 곡선
\end{enumerate}

\section{기대 결과와 성공 기준}
\subsection{사전등록 성공 기준}
\begin{enumerate}
  \item B2(+20\%) cost-matched 조건에서 macro $\Delta F1>0$이며 95\% CI 하한 $>0$
  \item Error Localization F1 유의 개선
  \item Full-Rerun 대비 Cost-per-Gain 우위
  \item Pareto 상 BCR이 최소 2개 tier에서 비지배(non-dominated)
\end{enumerate}
\begin{itemize}
  \item EM +3p는 exploratory 목표치로 유지하고, 주성공 판정은 Primary endpoint 기준으로 수행한다.
\end{itemize}

\subsection{참고 기대치(시뮬레이션 기반)}
\begin{enumerate}
  \item 보수적: EM +2p 내외
  \item 현실적: EM +4\textasciitilde{}6p, F1 +2\textasciitilde{}4p
  \item 낙관적: EM +7p 이상
\end{enumerate}

\section{리스크 및 대응}
\begin{enumerate}
  \item 반사실적 검증비 과다 $\rightarrow$ gate 강화, 조기종료, $n_{\max}$ 제한
  \item 개선폭 미미 $\rightarrow$ 정확도 단독이 아닌 Pareto/cost-per-gain 중심 기여 강화
  \item 재현성 저하 $\rightarrow$ config/seed 고정, 로그 표준화, 실행 스크립트 공개
\end{enumerate}


\section{선행/관련 연구 요약}
\begin{enumerate}
  \item \textbf{Adaptive-RAG}: 질의 복잡도 기반 전략 전환으로 효율성은 확보하나, 실패 hop의 인과적 기여도 추정과 국소 복구를 직접적으로 다루지는 않는다.
  \item \textbf{HopRAG/PropRAG/HydraRAG}: 멀티홉 경로 품질 및 멀티소스 결합을 강화하지만, 복구 호출 자체를 예산 제약 하에서 최적화하는 정책은 제한적이다.
  \item \textbf{ERL 계열}: 오류 위치화 및 검증 신호 측면의 기여는 크지만, RAG 복구 정책과 비용-성능 동시 최적화로의 연결은 상대적으로 약하다.
  \item \textbf{본 연구의 차별성}: 오류 위치화, 인과 점수(LCIS), 예산 기반 의사결정을 단일 정책으로 통합하여 정확도-근거품질-비용을 동시 최적화한다.
\end{enumerate}

\section{결론}
BCR은 멀티홉 RAG에서 모든 hop을 동일 강도로 검증하는 대신, 오류 가능성이 높은 hop에 한정해 반사실 검증과 국소 복구를 수행한다. 이를 통해 불필요한 호출을 억제하면서 정확도와 근거 체인 신뢰성을 동시에 개선하는 것을 목표로 한다. 본 연구의 핵심 기여는 단순 복구 기법 제안이 아니라, \textbf{복구 대상을 예산 제약 하에서 최적화하는 정책적 의사결정}에 있다.

\end{document}
